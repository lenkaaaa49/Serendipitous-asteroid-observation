\chapter{Introduction}
 
In this manual, the code made during the internship in SRON, is described and explained. The code's functionality is to determine which asteroids are in the field of view of the specified instrument at the specified time. The asteroids found and their physical characteristics, as known from the JPL HORIZONS System \cite{hor}, are saved in MySQL database. Using these information, the expected brightness (thermal emission) of each asteroid is calculated and saved in the database as well.\\

This tool uses an input table, Chapter \ref{chap:1}, with data about the desired date, instrument and field of view to run. The tool has 4 functions. First one, goes over all the rows in the input table and runs the code for the ones which have not been updated yet. For the second function, a date is specified. If the data in the MySQL table were updated before this date, the code is run for them again. Third one, updates the whole input table, creates new tables and updates all the old ones. Last, is similar to third, but a date is specified and the code creates new tables but updates only the outdated ones. \\

The principle of the tool is fairly simple. The inputs from the input table are taken and fed in the 
Asteroid \& Comet Field-of-View Search \cite{ispy} made by JPL Solar System Dynamics Group. This allows the user to identify known comets or asteroids that are present (and may be visible) in the instrument's field of view. This system returns a table with the objects found and some of their characteristics, such as their position (RA,DEC) or the uncertainty in the positional prediction. The dynamics are from a high-precision n-body numerical integration with perturbations of other objects. The catalogued asteroids data are taken from the JPL HORIZONS System \cite{hor} and as new asteroids are detected daily and old data are updated, it is useful to rerun the code every few weeks.
These information from the Asteroid \& Comet Field-of-View Search System are then imported by the code into MySQL tables. Each table has a specific name, so called observation id, which specifies the observation date, instrument and field of view. More data from the JPL HORIZONS System are imported into the table as well and then these are used to calculate the expected brightness (thermal emission) of the asteroid at different wavelengths. This is of course saved in the MySQL table as well. \\

After the database with tables for each observation id is created, it is possible with the help of MySQL \cite{mysql} to examine the data. The user can compare different tables together, or inspect in how many tables a specific asteroid appears. In general, this tool can be useful when interpreting data from an instrument. The database can be used to see which asteroids could be in the field of view and what brightness they should have at the specified wavelength according to what is known about them. Knowing this, and having the actual data from the instrument, the information for each of the asteroids can be updated to present better physical characteristics. \\

This manual consists of 4 chapters. In Chapter \ref{chap:2}, the tool requirements are stated. The inputs for the code are discussed in Chapter \ref{chap:1} and Chapter \ref{chap:3} gives a short overview of how to run the code and what are the expected outputs. %In Chapter \ref{chap:4} the methodology of the code is discussed.



